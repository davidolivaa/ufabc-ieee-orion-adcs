\documentclass[a4paper, 12pt]{article}
\usepackage[utf8]{inputenc}
\usepackage[brazil]{babel} 
\usepackage[lmargin=3cm, tmargin=3cm, rmargin=2cm, bmargin=2cm]{geometry} 
\usepackage[T1]{fontenc} 
\usepackage{amsmath, amsthm, amsfonts, amssymb, dsfont, mathtools} 
\usepackage{blindtext}
\usepackage{graphicx}

\begin{document}
\maketitle

\begin{figure}[ht]
    \centering
    \includegraphics[width=5cm]{ tesla.png} %insere a figura no texto
    \caption{Logomarca Tesla}
    \label{Fig01} %permite marcar a imagem
\end{figure}

\begin{figure}[ht]
    \centering
    \includegraphics[width=5cm]{ SpaceX-Logo.png} %insere a figura no texto
    \caption{Logomarca SpaceX}
    \label{Fig02} %permite marcar a imagem
\end{figure}

%com ref podemos referenciar a imagem marcada
Figura \ref{Fig01} é a logomarca da Tesla, empresa do Elon Musk

Ambas as empresas da figura \ref{Fig01} e \ref{Fig02} sao inovadoras


% Estrutura do ambiente \minipage para quatro figuras:
\begin{figure}
    \begin{minipage}[!]{0.50\linewidth}
    \includegraphics[\width=\linewidth]{tesla}\\ \textbf{ a) Tesla}
    \end{minipage}
    \begin{minipage}[!]{0.50\linewidth}
    \includegraphics[\width=\linewidth]{tesla}\\ \textbf{ b) Tesla}
    \end{minipage}
    \begin{minipage}[!]{0.50\linewidth}
    \includegraphics[\width=\linewidth]{tesla}\\ \textbf{ c) Tesla}
    \end{minipage}
    \begin{minipage}[!]{0.50\linewidth}
    \includegraphics[\width=\linewidth]{tesla}\\ \textbf{ d) Tesla}
    \end{minipage}
    \caption{Figuras lado a lado}
    \label{fig03}
\end{figure}


\end{document}