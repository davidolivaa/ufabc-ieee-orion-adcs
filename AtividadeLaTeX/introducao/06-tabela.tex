\documentclass[a4paper,12pt]{article}
\usepackage[utf8]{inputenc} % Pacote para acentuação
\usepackage[brazil]{babel} % Para quem vai escrever em português brasileiro
\usepackage[lmargin=3cm,tmargin=3cm,rmargin=2cm,bmargin=2cm]{geometry} %Formato que lembra a ABNT
\usepackage{amsmath,amsthm,amsfonts,amssymb,dsfont,mathtools} %pacotes matemáticos
\usepackage{graphicx}
\begin{document}

\begin{table}
    \centering 
    \begin{tabular}{l|c|c|c}\hline
    \textbf{Matérias}      & \textbf{Prova 1} & \textbf{Prova 2} & \textbf{Nota Final} \\\hline
    Cálculo 1              &  7,5 & 7,0 & $NFC_1$\\
    Álgebra Linear         &  8,5 & 6,0 & NFA\\
    Química 1              &  6,5 & 6,0 & $NFQ_1$\\
    Intro. às Engenharias  &  9,5 & 10,0& NFI\\
    \end{tabular}\\
    \caption{Notas nas disciplinas}
    \label{Tab01}
\end{table}

\begin{table}[!]
    \centering 
    \begin{tabular}{l|c|c|c}\hline
    \textbf{Matérias}      & \textbf{Prova 1} & \textbf{Prova 2} & \textbf{Nota Final} \\\hline
    Cálculo 1              &  7,5 & 7,0 & $NFC_1$\\
    Álgebra Linear         &  8,5 & 6,5 & NFA\\
    Química 1              &  6,5 & 5,0 & $NFQ_1$\\
    Intro. às Engenharias  &  9,0 & 10,0& NFI\\
    \end{tabular}\\
    \caption{Notas atualizadas nas disciplinas}
    \label{Tab02}
\end{table}

A tabela \ref{Tab02} esta mais atualizada que a tabela \ref{Tab01}

\end{document}